\message{ !name(trackeR.Rnw)}\documentclass[nojss]{jss}
\usepackage{amsmath,amssymb,amsfonts,thumbpdf}
\usepackage{dsfont}
%\usepackage{pstricks}
\usepackage{tikz}
\usepackage{setspace}
\usepackage{subcaption}
\usepackage{color}

%% additional commands
\newcommand{\squote}[1]{`{#1}'}
\newcommand{\dquote}[1]{``{#1}''}
\newcommand{\fct}[1]{{\texttt{#1()}}}
\newcommand{\class}[1]{\dquote{\texttt{#1}}}
%% for internal use
\newcommand{\fixme}[1]{\emph{\marginpar{FIXME} (#1)}}
\newcommand{\readme}[1]{\emph{\marginpar{README} (#1)}}
\newcommand{\iknote}[1]{{\color{blue} \bf [IK: #1] \marginpar{NOTE}}}
\newcommand{\hfnote}[1]{{\color{orange} \bf [HF: #1] \marginpar{NOTE}}}

\author{Hannah Frick\\University College London \And
	Ioannis Kosmidis\\University of Warwick \\ The Alan Turing Institute}
\Plainauthor{Hannah Frick, Ioannis Kosmidis}

\title{\pkg{trackeR}: Infrastructure for Running, Cycling and Swimming Data from
  GPS-Enabled Tracking Devices in \proglang{R}}
\Plaintitle{trackeR: Infrastructure for Running, Cycling and Swimming Data from
  GPS-Enabled Tracking Devices in R}
\Shorttitle{\pkg{trackeR}: Infrastructure for Running, Cycling and Swimming Data in
\proglang{R}}

\Keywords{sports, tracking, work capacity, running, cycling, distribution profiles}
\Plainkeywords{sports, tracking, work capacity, running, cycling, distribution profiles}


\Abstract{

  {\color{blue} This introduction to the \proglang{R} package
    \pkg{trackeR} is a modified version of
    \cite{trackeR:Frick+Kosmidis:2017}, published in the \emph{Journal
      of Statistical Software}. The majority of changes and additions
    in the text are due to work by Ioannis Kosmidis, which involved
    the complete rewrite of most of the \pkg{trackeR} codebase at
    version 1.1.0, in order to make the package faster across the
    board and include support of all running, cycling and swimming and
    sport-specific units and variables, fix bugs, and implement new
    features and enhancements.}

  The use of GPS-enabled tracking devices and heart rate
  monitors is becoming increasingly common in sports and fitness
  activities. The \pkg{trackeR} package aims to fill the gap
  between the routine collection of data from such devices and their
  analyses in \proglang{R}.
  The package provides methods to import tracking data into data structures
  which preserve units of measurement and are organised in sessions.
  The package implements core infrastructure for
  relevant summaries and visualisations, as well as support for handling units
  of measurement. There are also methods for relevant analytic tools such
  as time spent in zones, work capacity above critical power (known as
  $W'$), and distribution and concentration profiles. A case study
  illustrates how the latter can be used to summarise the information
  from training sessions and use it in more advanced statistical
  analyses.}

\Address{
  Hannah Frick\\
  Department of Statistical Science\\
  University College London\\
  Gower Street\\
  London, WC1E 6BT\\
  United Kingdom\\
  E-mail: \email{hannah.frick@gmail.com} \\
  URL: \url{http://www.ucl.ac.uk/~ucakhfr/} \bigskip \\
  %
  Ioannis Kosmidis \\
  Department of Statistics \\
  University of Warwick \\
  Gibbet Hill Road \\
  Coventry, CV4 7AL \\
  United Kingdom \\ \smallskip
  E-mail: \email{ioannis.kosmidis@warwick.ac.uk} \\
  URL: \url{http://www.ikosmidis.com}
}

%% Sweave/vignette information and metadata
%% need no \usepackage{Sweave}
\SweaveOpts{engine = R, eps = FALSE, keep.source = TRUE}
%\VignetteIndexEntry{Infrastructure for Running, Cycling and Swimming Data}
%\VignetteDepends{trackeR}
%\VignetteKeywords{sports, tracking, W', running, cycling, swimming distribution profiles, R}
%\VignettePackage{trackeR}

<<preliminaries, echo=FALSE, results=hide>>=
options(width = 70, prompt = "R> ", continue = "+  ", useFancyQuotes = FALSE)
library("trackeR")
library("ggplot2")
data("runs", package = "trackeR")
set.seed(403)
cache <- FALSE
@

\begin{document}

\message{ !name(trackeR.Rnw) !offset(3298) }
\section{Concluding remarks} \label{sec:outro}

% The \pkg{trackeR} package provides classes and methods for importing and
% handling the irregularly sampled spatio-temporal data that arise from
% GPS-enabled tracking devices in sports. Currently it supports TCX and db3
% formats, but the package structure allows the easy inclusion of other
% formats. After careful processing, the data are stored in session-based,
% unit-aware, and operation-aware objects of class \class{trackeRdata}. Associated
% methods handle conversion between units of measurement, calculation of relevant
% summaries and the visualisation of tracking data. Time in zones, work capacity,
% and distribution and concentration profiles can also be readily calculated and
% used for further statistical analyses. An example of such analysis is presented
% in Section~\ref{sec:casestudy}, where functional principal components analysis
% is used to characterise the apparent variability between speed concentration
% profiles.

% Infrastructure for irregularly sampled spatio-temporal data in general is also
% provided in the \pkg{trajectories} package \citep{trackeR:pkg:trajectories},
% which is developed around the \class{STIDF} class of the \pkg{spacetime} package
% \citep{trackeR:Pebesma:2012}. In order to provide users with access to an as
% wide range of data-analytic methods as is possible, we aim to provide a sensible
% coercion function from \class{trackeRdata} to \class{STIDF}. One of the
% challenges we are facing is that the \class{STIDF} class does not
% accommodate missing values in positional or temporal information, which is
% commonly the case in data from GPS-enabled tracking devices (e.g., sequences of
% missing values in the positional data because the GPS signal is temporarily
% lost). Other packages that offer tools for spatio-temporal data include
% \pkg{adehabtitatLT} \citep{trackeR:Calenge:2006}, \pkg{trip}
% \citep{trackeR:pkg:trip} and \pkg{move} \citep{trackeR:pkg:move}. The main focus
% of those packages is on animal tracking (e.g., estimation of habitat choices)
% and they are not directly suitable for tracking the various aspects of athlete activity.

% Areas for further development include the extension of the reading capabilities
% to more file formats and the implementation of more analytic tools, such as
% record power profiles \citep{trackeR:Pinot+Grappe:2011}. The case study in
% Section~\ref{sec:casestudy} draws some links to functional data analysis which
% is another area of further exploration for analytic tools for the package.

%\newpage \clearpage
\section*{Acknowledgements}

We are thankful to Victoria Downie, Andy Hudson, Louis Passfield, Ben
Rosenblatt, and Achim Zeileis for helpful feedback and discussions as well as
providing the data that are used for the illustrations and examples in the package.

\pagebreak
\clearpage

\bibliography{trackeR}


%%check journal and publisher names according to \url{http://www.jstatsoft.org/style}

%% Journals:
%%     The American Statistician (not: American Statistician)
%%     The Annals of Statistics (not: Annals of Statistics)
%%     Journal of the Royal Statistical Society B (not: Journal of the Royal Statistical Society, Series B)

%% Publishers:
%%     Springer-Verlag (not: Springer)
%%     John Wiley & Sons (not: Wiley, John Wiley & Sons Inc.)


%\newpage
\begin{appendix}

\section[Replenishment of W']{Replenishment of $W'$}
\label{sec:appendix}

Assuming that power is constant, the solution of the differential equation
describing the rate of replenishment in Equation~\ref{eq:recharge:rate} with respect
to $W'(t)$ gives
%
\begin{equation}
  \label{eq:recharge:2}
  1 - \frac{W'(t)}{W'_0} = \exp \left( \frac{P- CP}{W'_0}  t + D / W'_0 \right)
  \, .
\end{equation}
%
Using Equation~\ref{eq:recharge:2} over an interval $[t_{i-1}, t_i)$ of constant
power gives
%
\begin{equation*}
  1 - \frac{W'(t_i)}{W'_0} = \exp \left( \frac{P_i - CP}{W'_0}  (t_i -
    t_{i-1}) \right) \left( 1 - \frac{W'(t_{i-1})}{W'_0} \right) \, .
\end{equation*}
%
Hence, $W'(t_i)$ can be expressed in terms of $W'(t_{i-1})$ as
%
\begin{equation*}
   W'(t_i) = W'_0 - \left( W'_0 - W'(t_{i-1}) \right) \exp \left( \frac{P_i -
      CP}{W'_0}  (t_i - t_{i-1}) \right) \, .
\end{equation*}
%


%% , via the substitution rule for integrals in the
%% step from Equation~\ref{eq:recharge:substitution1}
%% to~\ref{eq:recharge:substitution2}, gives the following
%% %
%% \begin{align}
%%   \int{ \frac{d}{dt} W'(t) \frac{1}{1 -  W'(t)/W'_0} } \, dt &= \int{ (CP
%%     - P)} \, dt \label{eq:recharge:substitution1} \\
%%   \int{ \frac{1}{1 -  W'(t)/W'_0} } \, d W'(t) & = \int{(CP - P)} \, dt \label{eq:recharge:substitution2}\\
%%   \log \left( 1 - \frac{W'(t)}{W'_0} \right) (-W'_0) &= (CP- P)
%%   t + D  \nonumber \\
%%   1 - \frac{W'(t)}{W'_0} &= \exp \left( \frac{P- CP}{W'_0}  t + c \right)  \label{eq:recharge:2}
%% \end{align}
%% %
%% with $c = D / W'_0$.
%% Note that power is again assumed to be constant, as it is in the case of
%% depletion of $W'$. Applying this to an interval $[t_{i-1}, t_i)$ and setting
%% Equation~\ref{eq:recharge:2} for time points $t_{i-1}$ and $t_i$ in relation to
%% each other allows expressing $W'(t_i)$ in terms of $W'(t_{i-1})$:
%% %
%% \begin{align*}
%%   \frac{1 - \frac{W'(t_i)}{W'_0}}{1 - \frac{W'(t_{i-1})}{W'_0}} &= \frac{\exp
%%     \left( \frac{P_i - CP}{W'_0}  t_i + c \right) }{\exp \left( \frac{P_i -
%%         CP}{W'_0}  t_{i-1} + c \right) } \\
%%   1 - \frac{W'(t_i)}{W'_0} &= \exp \left( \frac{P_i - CP}{W'_0}  (t_i -
%%     t_{i-1}) \right) \left( 1 - \frac{W'(t_{i-1})}{W'_0} \right) \\
%%   W'_0 - W'(t_i) &= \exp \left( \frac{P_i - CP}{W'_0}  (t_i -
%%     t_{i-1}) \right) \left( W'_0 - W'(t_{i-1}) \right) \\
%%   W'(t_i) &= W'_0 - \left( W'_0 - W'(t_{i-1}) \right) \exp \left( \frac{P_i -
%%       CP}{W'_0}  (t_i - t_{i-1}) \right) \, .
%% \end{align*}
%% %


\end{appendix}



\message{ !name(trackeR.Rnw) !offset(3295) }

\end{document}
